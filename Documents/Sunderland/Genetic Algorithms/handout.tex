\documentstyle[12pt]{article}
\title{{Final Year Project\\
Genetic Algorithms\\
Final Presentation Handout}}
\author{Andreas Grimm}
\frenchspacing
\def\bs{\char'134 } % backslash in \tt font.
\pagestyle{headings}
\parindent0em \parskip1.5ex plus0.5ex minus 0.5ex
\renewcommand{\baselinestretch}{1.5}
\begin{document}
\maketitle
\tableofcontents
\newpage
\section{Objectives}
The following objectives should be solved in my final year project:\marginpar{Folie :\\Objectives}
\begin{enumerate}
\item Constructing an experimental environment for users of Genetic Algorithms.\\
Learning the use of Genetic Algorithms is very complicated for newcomers in this
area. Aim in this project is the construction of a number of libraries which
will help a potential user to construct genetic systems.
\item The environment should be programmed in Ada.\\
Basic idea is that the programming should be in an object orientated language which could be used
by a greater number of users. The computer language, which is best known by the
students and also most portable between different system is Ada. An alternative
would be the language C++, but there are not enough platforms availiable.
\item The packages should represent the different levels of Genetic Algorithms.\\
A first package cover simple Genetic Algorithms as described by Goldberg\cite{Gol89}.
The second package is an extension of the first also covering the advanced mechanims.
A third package is the implementation for two dimensional individuals.
\item The report should help a newcomer to learn the use of Genetic Algorithms.\\
The report has two parts. The first part is a short introduction in the theory
of Genetic Algorithms, the second part is a case study on an existing problem. 
\end{enumerate}
\section{Summary on Genetic Algorithms}
A Genetic Algorithm is a special method of a non numerical optimization, using
the mechanisms of the evolution to optimize variables of a given problem. Over
the last millions and millions of years the nature proceeded an optimization
of a system which is so complex that it would be even impossible to discribe it.

The method of Genetic Algorithms was first used as a method to build models of\marginpar{Historie\\evtl. ausbauen}
the behaviour of spezies in nature, after a certain time the interest changed.
The first publication which reaches a greater audience was made by A.K.Dewdney\marginpar{evtl. zitieren}
in the Scientific American, November 1985. But special papers were issued in
the late 1960's by J.Holland at the University of Michigan, L.Fogel, A.Owens
et al. The mechanisms used today where first mentioned on a conference at Carnegie-
Mellon University in this time.

But until today the method of Genetic Algorithms are almost unknown and only used
in very specific problems. The best known area where these methods are in use
is the optimization of neuronal networks, here special researches are made at
the California Institute of Technology.

Only a very few universities and polytechnics work in this area in Europe. The
institutes I know are
\begin{itemize}
  \item the University of Nijmegen in the Netherlands. Here a team of scientists
tries to identify the function of some genes in the human DNA with the help of
Genetic Algorithms.
  \item Dortmund University, working in the area of high speed processing of
Genetic Algorithms with the help of transputers.
  \item Cologne Polytechnic researching areas of practical use of Genetic Algorithms
specially in the area machine learning and knowledge based systems.
\end{itemize}
As a fact, the mechanisms used in this field are well known, on the other hand
there is no knowledge why the systems work in the way they actually work.\marginpar{Folie :\\Zyklus}

Genetic Algorithms emulate the circle of life in nature. Individuals are born,\marginpar{ausbauen\\freihand\\evtl. 1.Kapitel}
they have to prove themselves in an environment, they produce a new generation
and they disappear. To make this process simpler the steps are made at one time,
all individuals are born, they prove their fitness and they reproduce at the
same time.

The initial process of birth is done by a random number generator, which constructs
the individuals by building strings or matrices of the numbers 0 and 1. These
first individuals are checked and measured by a fitness function. This function
returns a value for each individual, called the fitness. The fitness of the
single individual is a direct measurement for the probability, that the single
individual will reproduce its genetic material in the process to develop a new
generation.

The development of a new generation is done in three steps. The first step produces
a basis material of strings. The material is choosen from the old generation,
depending on the fitness and a random function. This combination is mentioned
in literature as the roulette wheel selection. It means, that every inidividual
gets a segment on a imaginary roulette wheel, the size of each segment depends
on the fitness of each individual. By turning the roulette wheel $n$- times
a number of inidividuals is choosen.\marginpar{Examples\\Goldberg ?}

The second step is called the crossover. The individuals choosen before are
paired, the genetic string is splitted at a random point and the information
is crossed between the partners.

The third and last step is the mutation, simulary to the mutation in nature.
This is the change of genes without a special reason, just by random.

The new developed generation is checked again and the process starts again. It
runs over many generations and should be stopped by the user after a certain
number of generations passed or a certain optimization is reached. This
simple process is the basis of all further Genetic Algorithms and shown in figure
\ref{figure1}. Of course by developing more complex problems the mechanisms
must be changed, too.
\begin{figure}
  \unitlength1.0cm
  \begin{picture}(15,15)
  \end{picture}
\caption[The Cycle of Genetic Algorithms]{The Cycle of Genetic Algorithms}
\label{figure1}
\end{figure}
\section{Progress}
The progress of the project is shown in figure \ref{figure2}. As can be seen,\marginpar{Folie :\\Progress}
the development of the system starts in October '91 by sorting the sources and
checking the ability of the language to support the development. Here at first
two problems occured :
\begin{enumerate}
\item I had to learn the computer language in a very short time. Ada is teached
in the third year, but it was necessary to start the programming before the lectures
finished the lessons on Ada.
\item By inspecting the language I saw, that the language does not support a
random number generator. I solved this problem by porting the function from
the language C, published in a book on numerical mathematics\cite{Pre89}.
\end{enumerate}
After these first preperation were finished I planned the modules until they\marginpar{Folie :\\Packages}
had the form they have today. The implementation and testing was done from November
'91 until mid January'92. Since then I write the report, which is expected to
be finished at the end of this term.

The packages will be available on three platforms :
\begin{itemize}
\item on the Apollo Workstations
\item on MS-Dos Computers with an Ada compiler
\item on Unix System V/3.2 with Meridian Ada compiler
\end{itemize}

The report is written with the \TeX \  wordprocessor, using the \LaTeX \  macro package.
\begin{figure}
  \unitlength1.0cm
  \begin{picture}(15,15)
  \end{picture}
\caption[The progress of the project]{The progress of the project}
\label{figure2}
\end{figure}
\begin{thebibliography}{99}
\addtocontents{toc}{Bibliography}
\bibitem[Gol89]{Gol89}D.E.Goldberg: Genetic Algorithms in Search,
                      Optimization and Machine Learning\\
                      Ann Arbor, 1989

\bibitem[Pre89]{Pre89}W.H.Press, B.P.Flannery, et al.:Numerical Recipes\\
                      The Art of Scientific Computing\\
                      (C Version)\\
                      Cambridge, 1989
\end{thebibliography}
\end{document}
