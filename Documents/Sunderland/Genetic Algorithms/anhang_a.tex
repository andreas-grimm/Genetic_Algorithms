This is the package for the use of simple genetic algorithms. It is absolutly
different from the PASCAL program developed by Goldberg\cite{Gol89}. It tries
to optimize the processes of the elements of genetic algorithms and also should
be a skeleton for the development of other, more specific genetic algorithms.
I assume that the mechanisms of the programming language ADA are known, so I
do not explain the syntax of it.
\section{The Header of the Package}
The following program is the header of the package. To use the package a programmer
only needs to know this part of it. Because it is so essential it is better
documented than the body.
\begin{verbatim}
pragma LIST(ON);

with text_io; use text_io;
with random; use random;
with float_io; use float_io;
with integer_io; use integer_io;
\end{verbatim}
These first used package are part of standart ADA environment. It is possible,
that in some implementations the names are different. I used the names of the
ADA implementation for the Apollo workstations. The package {\tt random} is part
of this report and not a standart of ADA.
\begin{verbatim}
generic
\end{verbatim}
The following parameters must be defined by instantiation of the package :
\begin{itemize}
  \item {\tt MaxAllele}: The number of genes\footnote{or alleles} in an individual
  \item {\tt MaxPopSize}: The number of individuals in the population
\end{itemize}
\begin{verbatim}
   MaxAllele            : integer;
   MaxPopSize           : integer;

package Simple_Genetic_Algorithms is
\end{verbatim}
The first definition in the package are the most important types for genetic
algorithms. This type build the interface of the program and the SGA package.
The used names find their equivalent in the book of Goldberg\cite{Gol89}.
\begin{verbatim}
  type Individuum       is array (1..MaxAllele) of integer range 0..1;
  type Population_Array is array (1..MaxPopSize) of Individuum;
  type Fitness_Array    is array (1..MaxPopSize) of float;
\end{verbatim}
The first function initializes the population. Because there is no special rule
an individual has to be initialized the procedure can be handled as a black
box, creating individuals without information how this is done.
\begin{verbatim}
  procedure Init_Population (Population : in out Population_Array);
\end{verbatim}
Next is the definition of the elementary functions of SGAs :
  \begin{enumerate}
  \item The roulette wheel,
  \item the crossover and
  \item the mutation.
  \end{enumeration}
First defined function is the {\tt roulette\_wheel}. This function produces
the roulette wheel selection. Parameters of this function are :
  \begin{itemize}
  \item {\tt Population} is the sum of all individuals.
  \item {\tt Fitness} is an array in which for all individuals is their specific
  fitness.
  \end{itemize}
\begin{verbatim}
  procedure Roulette_Wheel (Population : in out Population_Array;
                            Fitness    : in     Fitness_Array);
\end{verbatim}
The procedure {\tt Crossover} works on the population and returns the new
generation after the crossover step.
\begin{verbatim}
  procedure Crossover      (Population : in out Population_Array);
\end{verbatim}
The last of the three steps is the mutation. The procedure {\tt Mutation} is
an implementation of this step. The parameters of this procedure is again the
population and the quote of mutation, which is a floating point variable $< 1$.
\begin{verbatim}
  procedure Mutation       (Population : in out Population_Array;
                            Mutationquote: in   float);
\end{verbatim}
The last procedure is a simple output procedure. This procedure can be replaced
by the programmer without changing the core of the genetic algorithms. Its output
is a tabular with four columns :
  \begin{enumerate}
  \item the individuals,
  \item its percentage of fitness,
  \item the integer value the individual does represent and
  \item the result of the calculation done with the individual
  \end{enumerate}
\begin{verbatim}
  procedure Simple_Display(Generation : integer;
                           Population : in Population_Array;
                           Fitness    : in Fitness_Array;
                           Org_Value  : in Fitness_Array;
                           Calc_Value : in Fitness_Array);

end Simple_Genetic_Algorithms;
\end{verbatim}
This ends the header of this package.
\section{The Body of the Package}
The package body is normally closed as a secret. The mechanisms of ADA do not
allow, that anybody can inspect the following lines or uses the variables defined
inside.
\begin{verbatim}
package body Simple_Genetic_Algorithms is

  Best_Org_Value        : float := 0.0;
  Best_Calc_Value       : float := 0.0;
  Best_Generation       : integer := 0;
  Last_Gen_Accum_Org    : float := 0.0;
  Last_Gen_Accum_Calc   : float := 0.0;
\end{verbatim}
\subsubsection{The Initialization of the Population}
\begin{verbatim}
  procedure Init_Population (Population : in out Population_Array) is
    t, x                : integer;
    Single_Individuum   : Individuum;
  begin
    for t in Population'Range loop
      for x in Population(t)'Range loop
        Single_Individuum(x) := random_ND_integer(65535) mod 2;
        if (Single_Individuum(x) > 1) then
          Single_Individuum(x) := 1;
        end if;
      end loop;
      Population(t) := Single_Individuum;
    end loop;
  end Init_Population;
\end{verbatim}
\subsubsection{The Roulette Wheel Selection}
As seen above the first procedure is the roulette wheel selection.
\begin{verbatim}
  procedure Roulette_Wheel (Population : in out Population_Array;
                            Fitness    : in     Fitness_Array) is
    New_Population : Population_Array;
    Wheel_Length   : constant := 100;
    Wheel          : array (1..Wheel_Length) of integer;
    Position       : integer := 1;
    t              : integer;
    x              : integer;
    field          : integer;

  begin
\end{verbatim}
This is the initialization of the roulette wheel. Every individual gets
fields according its fitness, which must be than better then 1 \% of the accumulated
fitness in the system. The size of the roulette wheel will be less or equal
100 fields.
\begin{verbatim}
    for t in Population'Range loop
      if (Fitness(t) >= 1.0) then
        for x in Position..(Position+integer(Fitness(t))) loop
          if (x <= 100) then
            Wheel(x):=t;
          end if;
        end loop;
        Position := Position + integer(Fitness(t));
      end if;
    end loop;
\end{verbatim}
Start of the selection of the new generation.
\begin{verbatim}
    for t in Population'Range loop
      field := 101;
      while (field > 100) loop
        field := (random_ND_integer(65535) mod Position) + 1;
      end loop;
      New_Population(t) := Population(Wheel(field));
    end loop;
\end{verbatim}
Let the new generation take the place of the old one.
\begin{verbatim}
    for t in Population'Range loop
      Population(t) := New_Population(t);
    end loop;
\end{verbatim}
This finishes the procedure.
\begin{verbatim}
  end Roulette_Wheel;
\end{verbatim}
\subsubsection{The Crossover}
The crossover is the longest and most difficult part of the package.
\begin{verbatim}
  procedure Crossover (Population : in out Population_Array) is
    New_Population     : Population_Array;
    Partner            : array (Population'Range) of integer;
    x,y,z              : integer;
    filled             : boolean;
    Selected_Partner   : integer;
    Point_of_Crossover : integer;
    New_1st_Individuum : Individuum;
    New_2nd_Individuum : Individuum;
    Dummy              : Individuum;

  begin
\end{verbatim}
First it is necessary to initialize the fields for the relationships. This fields are
reserved for the information about the partners.
\begin{verbatim}
    for x in Partner'Range loop
      Partner(x) := Population'Length + 1;
    end loop;
\end{verbatim}
Each individual gets its partner.
\begin{verbatim}
    for x in 1..(Partner'Last / 2) loop
      Selected_Partner := (random_ND_integer(65535) mod (Population'Length / 2))
                           + (Population'Length / 2) + 1;
      loop
        filled := false;
        for y in 1..x loop
          if (Partner(y) = Selected_Partner) then
            filled := true;
            Selected_Partner :=(random_ND_integer(65535) mod (Population'Length
                                / 2)) +(Population'Length / 2);
            end if;
          end loop;
        exit when (filled = false);
      end loop;
      Partner(x) := Selected_Partner;
    end loop;
\end{verbatim}
Now the partners are selected. Now the procedure finds the point of crossover
for each pair and starts the change of the genetic information.
\begin{verbatim}
    for x in 1..(Partner'Last / 2) loop
      New_1st_Individuum := Population(x);
      New_2nd_Individuum := Population(Partner(x));
      Point_of_Crossover := (random_ND_integer(65535) mod Population(x)'Length);
      if (Point_of_Crossover > 0) then
        Dummy := New_1st_Individuum;
        for z in 1..Point_of_Crossover loop
          New_1st_Individuum(z) := New_2nd_Individuum(z);
          New_2nd_Individuum(z) := Dummy(z);
        end loop;
      end if;
      Population(x) := New_1st_Individuum;
      Population(Partner(x)) := New_2nd_Individuum;
    end loop;
  end Crossover;
\end{verbatim}
This ends this procedure. Important for all procedures is the fact that they
work independent from the size of the individuals and the population.
\subsubsection{The Mutation}
The mutation is a very short and simple procedure. It has only to check, if there are
any mutations and than choose the genes which must be changed.
\begin{verbatim}
  procedure Mutation       (Population : in out Population_Array;
                            Mutationquote: in   float) is
    Number_of_Mutations : integer;
    Selected            : integer;
    Point_of_Mutation   : integer;
    Mutant              : Individuum;
    Quote               : float;
    Correction          : float;

    begin
      Quote := Mutationquote;
\end{verbatim}
If the mutation quote is too small or too big a message for the user is printed.
The system also chooses a different medium quote.
\begin{verbatim}
      if (Mutationquote < 0.001) then
        put_line("Quote of Mutation too small ! Assumed new quote : 0.005 ");
        Quote := 0.005;
      elsif (Mutationquote > 0.999) then
        put_line("Quote of Mutation too big ! Assumed new quote : 0.005 ");
        Quote := 0.005;
      end if;
\end{verbatim}
The next step calculates the number of genes which have to mutate.
\begin{verbatim}
      Correction := 1000.0 / float(MaxPopSize);
      Number_of_Mutations := integer(Quote * float(MaxPopSize) * Correction);
\end{verbatim}
If the number of mutations is greater then 0, the individuals and their genes
are selected. Than the mutation is executed.
\begin{verbatim}
      if (Number_of_Mutations > 0) then
        Number_of_Mutations := random_ND_integer(65535) mod Number_of_Mutations;
      end if;
      if (Number_of_Mutations > 0) then
        for x in 1..Number_of_Mutations loop
          Selected := (random_ND_integer(65535) mod Population'Length) + 1;
          Point_of_Mutation := (random_ND_integer(65535) mod
                                 Population(Selected)'Length) + 1;
          Mutant := Population(Selected);
          if (Mutant(Point_of_Mutation) = 0) then
            Mutant(Point_of_Mutation) := 1;
          else
            Mutant(Point_of_Mutation) := 0;
          end if;
        Population(Selected) := Mutant;
        end loop;
      end if;
    end Mutation;
\end{verbatim}
\subsubsection{The Display Function}
The display function is not documentated.
\begin{verbatim}
  procedure Simple_Display(Generation : integer;
                           Population : in Population_Array;
                           Fitness    : in Fitness_Array;
                           Org_Value  : in Fitness_Array;
                           Calc_Value : in Fitness_Array) is
  Single_Allele         : integer;
  Element               : Individuum;
  This_Gen_Accum_Org    : float;
  This_Gen_Accum_Calc   : float;
  Change_of_Accum_Calc  : float;

  begin
    new_page;
    put ("Genetic Algorithms Experimental Environment: Simple Display ");
    put_line ("Procedure");
    new_line;
    put ("Generation :");
    put (Generation);
    new_line;
    for T in Population'Range loop
      Element := Population(T);
      for X in Element'Range loop
        Single_Allele := Element(X);
        put(Single_Allele,1);
      end loop;
    put(" | "); put(integer(Fitness(T)),4);
    put(" | "); put(integer(Org_Value(T)),4);
    This_Gen_Accum_Org := This_Gen_Accum_Org + Org_Value(T);
    put(" | "); put(integer(Calc_Value(T)),4);
    This_Gen_Accum_Calc := This_Gen_Accum_Calc + Calc_Value(T);
    if (Best_Calc_Value < Calc_Value(T)) then
      Best_Calc_Value := Calc_Value(T);
      Best_Generation := Generation;
    end if;
    new_line;
    end loop;
    Change_of_Accum_Calc := This_Gen_Accum_Calc - Last_Gen_Accum_Calc;
    put("Best Fitness :  "); put(integer(Best_Calc_Value));
    put(" in Generation : "); put(Best_Generation); new_line;
    put("Accum. Fitness: "); put(integer(This_Gen_Accum_Calc));
    put(" Last Generation : "); put(integer(Last_Gen_Accum_Calc));
    put(" Change : "); put(integer(Change_of_Accum_Calc)); new_line;
    Last_Gen_Accum_Calc := This_Gen_Accum_Calc;
  end Simple_Display;

end Simple_Genetic_Algorithms;
pragma LIST(OFF);
\end{verbatim}
