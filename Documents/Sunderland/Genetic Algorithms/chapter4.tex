\label{chap4beg}
In this chapter I want to demonstrate the use of the advanced genetic algorithms.
For this I use a well known problem of the copmuter science, the traveling salesman
problem. This is a very good example of an $\cal{NP}$-complete problem\cite{Hop88}.
This kind of problems cannot be solved by computer algorithms if the size of the
problem exceeds a certain number, because the number of calculations necessary
to solve these problems are growing by an exponential factor. Another example
of this class of functions is the Ackermann- function.

An interesting fact in the history of genetic algorithms is that in the first
years the researchers believed they could solve this kind of problems. So some
experiments were made, with the result that all attempts failed which means
that the computer calculate a nearly infinity time to get the result or that
the size of the population is so big that the result is automaticly a member
of the population.

There are different solutions using other methods but none of them can ignore
the fact that they will fail if the size is big enough.

The following program is not another try to solve this problem, but a demonstration
how to formulate a more complex problem as a genetic algorithm. For this I first
have a little revision on the problem itself.
\section{The Traveling Salesman}
A travelling salesman has to travel to a number of towns in order to sell his
goods to the local shops. For this it is important for him to enter each town
one time and to keep the distances between the different towns as small as possible.
Also he does not want to enter the same town a second time.

There are different strategies to solve that problem.
\begin{enumerate}
  \item Calculate all possible routes and choose the shortest. This means that
  the computer has to work on a huge number of possibilities. Using elementary
  statistics it is simple to prove that the computer has to calculate $n!$ different
  routes. So for 5 towns there are 120 different routes, but for 25 towns he
  has to calculate $1.55 \cdot 10^{25}$ different routes. I assume the user
  has a very quick computer which can calculate 10 routes a second this number
  means that the computer will have to work for approx. $5 \cdot 10^{16}$ years.
  If this imaginary computer would exist at the beginning of the universe it would
  be possible to get the results in some billion years. But raising the number of
  towns by ten the computer will work longer than this universe will exist.
  \item Using a simple hill-climbing algorithm. This method produces a huge
  number of calculations, too.
  \item There are also methods in the area of operation research discriped in
  literature. An example of this methods is the Knappsack-algorithm. But the
  biggest number of towns which is solved by this methods is under 50.
\end{enumerate}
To simplify the problem I assumed that between each town exists a road and the
distances between the towns are constant\footnote{i.e. the distance between A and
B is the same as the distance between B and A. In our normal life this is not
nescessarily true.}.
\section{Using Genetic Algorithms}
Transforming the existing problem into a form suitable for genetic algorithms
needs the definition of the two major components. These components are the individual
and the fitness function.
\subsection{Defining the Individuals}
The first task by using genetic algorithms to solve the traveling salesman problem
is to define the individual. By examining the problem it could be seen that the
main subject of this problem is the route the salesman has to take. So I am looking for
a representation of the route as a binary string.

First I coded the towns as numbers, defining a function which projects a town
on an integer. This is a basis to define a representation of the route. I can
define the route as the order of the towns. For example the number chain '132'
represents the route beginning at town no. 1 to town no. 3 and than going to
town no. 2. Another route is represented by the number chain '312' or '231' and
so on. As simply can be seen, each number can only be once in the string. If it
would occure a second time the system would contains loops. By defining the individuals
with a fixed length\footnote{In nature the DNA has a fixed length, too.} this
also excludes some towns from the route.

A special case is if the first town occurs in the string the second time. This
is called a short circuit. The travel ends before all towns are visited because
the salesman reaches his origin again.

So using this method to code the route as an individual I transfer the number
chain into a binary string. For this a transform each number of the chain into
its binary representation is made. It has to be made sure, that the binary numbers
have the same length, i.e. that they have the same number of digits. Writing
the digits in the string forms the individual I am looking for.

What makes the representation really difficult to use is the fact, that there
are more individuals which represent forbidden routes than there are individuals
which represent good and useful routes. As can be seen later in the experiments
this problems will grow to a dimension which makes the system nearly useless.

\subsection{Finding the Fitness-Function}
The fitness function is not too complicate to find. The aim on this quest is to
find the shortest route around all cities. So the first and best idea is that
the driven distance is a direct measurement for the route. By following the route
town by town a number of driven miles can be calculated. A route is better for the
salesman if its milage is smaller than the milage of the other routes.

So the problem in this specific exercise is not the fitness function but the representation
of the individual.
\section{Using the Package}
Trying to use the SGA package shows very quick, that this package is not suitable.
At several points I need the mechanisms of the advanced genetic algorithms :
  \begin{itemize}
  \item The individuals must not crossover at a randomly selected point. Doing
  this would distroy the whole information of the string. Only at the ends of the
  numbers is a crossover allowed.
  \item The crossover also uses the haploid crossover mecahnism.
  \item The mutation quote should be close to zero, any change of a single binary digit will
  destroy the whole individual.
  \end{itemize}
By using these mechanisms it should possible to solve the problem. But after a
few test runs it could be seen, that the number of bad individuals is huge and that
after a few generations all individuals are dead. So it is necessary to install
a repair algorithm, according to the section on handicapped individuals in the
last chapter.

The result of these {\bf Gedanken} is the following program.
\subsubsection{Initialization of the Package}
A generic package in Ada must be initalized before it can be used. The first
step in the program is the using of most of the standart packages.
\begin{verbatim}
with Text_io;
with iio; with fio;
with Advanced_Genetic_Algorithms;

procedure Traveling_Salesman is
-- Number of Allele in the String : 8
-- Population : 10 Strings
-- Number of Points to cut the String : 7
  package Genetic_Algorithms is new Advanced_Genetic_Algorithms(24,10,7);
\end{verbatim}
This line initializes the package for the {\tt Advanced\_Genetic\_Algorithms}.
The three parameters of initialization are
  \begin{enumerate}
  \item - The size of the individual, in this case it is the number of towns times three.
  I use eight towns for this demonstration, these can be coded by a three digit
  binary number. So the size of a individual is $8 \times 3 = 24$ bits.
  \item - The size of the population. This value is found by try and should be
  changed during the experiments.
  \item - The number of crossover points. I use eight towns, so there are $8 - 1 = 7$
  crossover points.
  \end{enumerate}
\begin{verbatim}
  use Genetic_Algorithms;
  use Text_io; use iio; use fio;
\end{verbatim}
\subsubsection{Definition of the elementary variables}
In the next lines I defined the nescessary variables and arrays. This is alway
a task for the programmer, because he is the only one who knows the semantic
contents the genetic algorithms are used in.
\begin{verbatim}
  No_of_Towns : constant integer := 8;
  type Route  is array (1..No_of_Towns) of integer;
  type Routes is array (1..80) of Route;
\end{verbatim}
The following declaration is the table to change a number to a town. For example
the number 1 represents London, the number 8 represents Dover. Also the distances
between the towns are defined in this program segment.
\begin{verbatim}
  Towns       : array (1..No_of_Towns) of string(1..10) :=
    ("London    ",
     "Edinburgh ",
     "Cardiff   ",
     "Newcastle ",
     "York      ",
     "Holyhead  ",
     "Thurso    ",
     "Dover     ");
  Distance_Array : array (1..No_of_Towns,1..No_of_Towns) of integer :=
   ((  0, 378, 159, 274, 193, 259, 651,  71),
    (378,   0, 385, 110, 194, 308, 278, 449),
    (159, 385,   0, 301, 240, 201, 650, 237),
    (274, 110, 301,   0,  84, 247, 384, 345),
    (193, 194, 240,  84,   0, 188, 462, 264),
    (259, 308, 201, 247, 188,   0, 587, 339),
    (651, 278, 650, 384, 462, 587,   0, 717),
    ( 71, 449, 237, 345, 264, 339, 717,   0));
\end{verbatim}
The table of the distances are taken from \cite{Col91}. Are following declarations
are standart and also used in the first example.
\begin{verbatim}
  Population     : Population_Array;
  Fitness        : Fitness_Array;
  Generation     : integer := 0;
  best_Generation: integer := 0;
  Value_Array    : Fitness_Array;
  Calc_Array     : Fitness_Array;
  Way            : Routes;
  Mutation_Quote : float := 0.05;
  Crossing_Dummy : Crossing_Points;
  best_Way       : Route;
  best_Value     : float := 99999.99;
  Roulette       : boolean := false;
\end{verbatim}
\subsubsection{The Output}
This first procedure is the output of the system, it shows the found routes of
the system. To keep the program small I only installed a very rough procedure, in
a real system the user interface should be more satisfying.
\begin{verbatim}
  procedure Show_Way (Way : in Route) is
    begin
    for t in 1..(Way'Last-1) loop
      put(Towns(Way(t))); put(" --> ");
      if ((t mod 4) = 0) then
        new_line;
        end if;
      end loop;
    put(Towns(Way(Way'Last)));
    end Show_Way;
\end{verbatim}
The next procedure changes the binary string into the number chain which represents
the route.
\begin{verbatim}
  procedure Change_String_to_Route (Genetic_String  : in     Individuum;
                                    Transformed_Way : in out Route) is
    x : integer := 1;
    begin
    for t in Transformed_Way'Range loop
      Transformed_Way(t) := Genetic_String(x) * 4 + Genetic_String(x+1) * 2
                                                  + Genetic_String(x+2) + 1;
      x := x + 3;
      end loop;
    end Change_String_to_Route;
\end{verbatim}
\subsubsection{The Fitness Function}
The following lines are the implementation of the fitness function. The fitness
function also contains the repair function to produce handicapped individuals.
Of course this way including the repair function is not a must, it can be done
differently and will work as good as this one.
\begin{verbatim}
  procedure Check_Fitness(Population  : in out Population_Array;
                          Way         : in out Routes;
                          Calc_Array  : in out Fitness_Array;
                          Fitness     : in out Fitness_Array) is
    Sum_of_Fitness  : float := 0.0;
    Sum_Distance    : float := 0.0;
    Number_of_Knots : integer := 0;
    Error_Coeff     : integer := 0;
    Double          : integer := 0;
\end{verbatim}
This function calculates the distance the salesman has to travel following the
route given into the function. If the route contains a short circuit the return
value is of maximum size.
\begin{verbatim}
    function Distance (Way : in Route) return float is
      Way_Gone     : float   := 0.0;
      Short_Cycle  : boolean := false;
      used_Towns   : array(1..No_of_Towns) of integer;
      Acculum_Dist : integer := 0;
      begin
      for t in used_Towns'Range loop
        used_Towns(t) := Way(t);
        end loop;
      for t in 1..(Way'Last-1) loop  -- former -1!
        for x in used_Towns'Range loop
          if ((used_Towns(x) = Way(t)) and (x /= t)) then
            Short_Cycle := true;
            Error_Coeff := Error_Coeff + 1;
            Double      := t;
            end if;
          end loop;
        used_Towns(t) := Way(t);
        Acculum_Dist := Acculum_Dist + Distance_Array(Way(t),Way(t+1));
        end loop;
      if (Short_Cycle = true) then
        return 99999.99;
      else
        return float(Acculum_Dist);
        end if;
      end Distance;
\end{verbatim}
The following procedure is the repair function. As can be seen, this function
corrects every route which has less then two errors.
\begin{verbatim}
    procedure Repair(Element : in out Individuum; Way : in Route) is
      missing      : integer := 0;
      First_Allele : integer := 0;
      in_it        : boolean := false;
      begin
      for t in 1..No_of_Towns loop
        for x in 1..No_of_Towns loop
          if (Way(x) = t) then
            in_it := true;
            end if;
          end loop;
        if (in_it = false) then
          missing := t;
          in_it := false;
          end if;
        end loop;
      First_Allele := (Double * 3) - 2;
      for t in First_Allele..(First_Allele+2) loop
        Element(t) := missing / 2;
        missing := missing mod 2;
        end loop;
      end Repair;
\end{verbatim}
The following procedure is the main part of the fitness function. It works as
follows: First the binary string will be converted to a route. This route is
checked and repaired if there exists a chance for this. After this a fitness
is calculated.
\begin{verbatim}
    begin
    for t in Population'Range loop
      Change_String_to_Route (Population(t),Way(t));
      Calc_Array(t) := Distance(Way(t));
      if (Error_Coeff = 1) then
        Repair(Population(t),Way(t));
        Change_String_to_Route (Population(t),Way(t));
        Calc_Array(t) := Distance(Way(t));
        end if;
      Error_Coeff := 0;
\end{verbatim}
If the individual is healthy it gets a distance of less than $99999.99$. Then
the individual joins in the reproduction and the length of the route is added
to the sum of distances.
\begin{verbatim}
      if (Calc_Array(t) /= 99999.99) then
        Sum_Distance := Sum_Distance + Calc_Array(t);
        end if;
      end loop;
\end{verbatim}
If the individual could not be repaired it gets the fitness of 0.
\begin{verbatim}
    for t in Population'Range loop
      if (Calc_Array(t) = 99999.99) then
        Fitness(t) := 0.0;
      else
        Fitness(t) := (Calc_Array(t) / Sum_Distance) * 100.0;
        end if;
      end loop;
    end Check_Fitness;
\end{verbatim}
\subsubsection{The Main Program}
As can be seen the main program is very simple again. It starts with the elementary
initialization of the population. This is done with the random number generator
of the package. After this is done the program enters the loop as in the first
example.

Some more lines are included to the main structure to produce additional output
and information for the user. This is a good hint for everybody who wants to program
genetic algorithms himself. Additional informations are necessary to find were the
problems in the genetic system occur.
\begin{verbatim}
  begin
    Init_Population(Population);
    best_Way := Way(1);
    loop
      Check_Fitness(Population, Way, Calc_Array, Fitness);
      Generation := Generation + 1;
      put("Generation : "); put(Generation); new_line;
\end{verbatim}
The next lines select the output: If a better route is found, it is remembered.
\begin{verbatim}
      for t in Population'Range loop
        if (best_Value > Calc_Array(t)) then
          best_Value := Calc_Array(t);
          best_Way   := Way(t);
          best_Generation := Generation;
          end if;
\end{verbatim}
All practical routes are displayed. This lines should be cancelled if the displayed
information hides the important information of the system.
\begin{verbatim}
        if (Fitness(t) > 0.0) then
          Show_Way(Way(t)); put(integer(Calc_Array(t))); new_line;
          end if;
        end loop;
\end{verbatim}
The best route found so far is displayed, too.
\begin{verbatim}
      put("best Value : Generation : "); put(best_Generation); new_line;
      Show_Way(best_Way); put(integer(best_Value)); new_line;
\end{verbatim}
If there are not enough individuals for the next generation, the roulette wheel
reproduction is cancelled. This might be nescessary in some cases.
\begin{verbatim}
      Roulette := false;
      for t in Fitness'Range loop
        if (Fitness(t) > 0.0) then
          Roulette := true;
          end if;
        end loop;
      if (Roulette = true) then
        Roulette_Wheel(Population,Fitness,0);
        end if;
\end{verbatim}
The rest of the procedure look quite familiar.
\begin{verbatim}
      Crossover(Population,haploid,randomize,Crossing_Dummy,0,0);
      Mutation(Population,Mutation_Quote,0);
      end loop;
  end Traveling_Salesman;
\end{verbatim}
\section{Additional Comments}
This system is, of course, a very small one. But working with a bigger system
would expand the code and would be very hard to understand, too. Surely not all
problems could be solved with this package, there are problems which require
a two dimensional system of genetic algorithms. Also there may be problems which
require different procedures. Genetic algorithms are in the beginning of the
use outside of the labors and the number of researched problems is not very
big. Most discussed problems are of a very theoretical nature and a lot of the
real world problems may look different.

This procedure is a hint how such a system may look alike, as written above
there is no rule how a genetic algorithms has to be.

A full description of the interface to the different packages is in the appendix,
as well as a full listing of all of the used packages.