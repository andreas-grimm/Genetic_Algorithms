The programming language has no random number generator included. But this
generator is vital for the use of genetic algorithms. The following package
is developed to include such a generator into the used packages. It depends
on a chapter in \cite{Pre88}. There a full documentation of the functions
could be found, the names used here are mostly identical to the names used in
this book.

The following module generates random integers and floating point numbers, which
are normal and exponential deviated.
\begin{verbatim}
-- Random number generators in Ada
-- Version 1.0
-- see: W.H.Press, B.P.Flannery, S.A.Teukolsky, W.T.Vetterling
--      Numerical Recipes in C
--      Cambidge, 1988
--      ISBN 0-521-35465-X

--      Chapter 7, Pages 204 - 241

with calendar;
with math_lib;

package random is
-- Function for normal (Gaussian) deviates
  function random_ND return float;
  function random_ND_integer(reach : integer) return integer;
-- Function for Exponential deviates
  function random_ED return float;
  function random_ED_integer(reach : integer) return integer;
end random;

package body random is
-- static variables for the function "ran1"
  M1            : integer := 259200;
  M2            : integer := 134456;
  M3            : integer := 243000;
  IA1           : integer := 7141;
  IA2           : integer := 8121;
  IA3           : integer := 4561;
  IC1           : integer := 54773;
  IC2           : integer := 28411;
  IC3           : integer := 51349;
  RM1           : float   := 0.0000038502;
  RM2           : float   := 0.0000074377;
  x             : integer := 737;
  ix1, ix2, ix3 : integer;
  r             : array (0..98) of float;
  iff           : integer := 0;
  initialized   : boolean := false;
  init_value    : integer := -31415;
-- static variables for the function "ran2"
  M             : integer := 714025;
  IA            : integer := 1366;
  IC            : integer := 150889;
-- static variables for the function "ran3"
  MBIG          : integer := 1000000000;
  MSEED         : integer := 161803398;
  MZ            : integer := 0;
  FAC           : float;
-- static variables for the function "random_ND"
  iset          : integer := 0;
  gset          : float   := 0.0;

  function ran1 return float is
    temp : float;
    j    : integer;
    begin
      if (initialized  = false) then
        ix1 := (IC1 - init_value) mod M1;
        ix1 := (IA1 * ix1 + IC1) mod M1;
        ix2 := ix1 mod M2;
        ix1 := (IA1 * ix1 + IC1) mod M2;
        for j in 1..97 loop
          ix1 := (IA1 * ix1 + IC1) mod M1;
          ix2 := (IA2 * ix2 + IC2) mod M2;
          r(j) := float(ix1 + ix2 * integer(RM2)) * RM1;
        end loop;
        initialized := true;
        end if;
      ix1 := (IA1 * ix1 + IC1) mod M1;
      ix2 := (IA2 * ix2 + IC2) mod M2;
      ix3 := (IA3 * ix3 + IC3) mod M3;
      j   := 1 + ((97 * ix3) / M3);
      temp := r(j);
      r(j) := float(ix1 + ix2 * integer(RM2)) * RM1;
      return temp;
    end ran1;

    function random_ND return float is
      use math_lib;
      fac, r, v1, v2    : float;
      temp              : float;
      begin
      if (iset = 0) then
        loop
          v1 := 2.0 * ran1 -1.0;
          v2 := 2.0 * ran1 -1.0;
          r  := v1 * v1 + v2 * v2;
          exit when (r >= 1.0);
        end loop;
        temp := ln(r)/r;
        if (temp < 0.0) then
          fac  := sqrt(-2.0 * temp);
        else
          fac  := sqrt(2.0 * temp);
        end if;
        gset := v1 * fac;
        iset := 1;
        temp := v2 * fac;
        return temp;
      else
        iset := 0;
        return gset;
      end if;
      end random_ND;

      function random_ND_integer(reach : integer) return integer is
        value           : integer := 0;
        minus_one       : integer := -1;
      begin
        value := integer(random_ND * float(reach));
        if (value < 0) then
          value := (value * minus_one);
        end if;
        return value;
      end random_ND_integer;

      function random_ED return float is
      use math_lib;
        dummy : float := 0.0;
      begin
        dummy := -1.0 * (ln(ran1)/ln(10.0));
        return dummy;
      end random_ED;

      function random_ED_integer(reach : integer) return integer is
        value           : integer := 0;
        minus_one       : integer := -1;
      begin
        value := integer(random_ED * float(reach)); 
        if (value < 0) then
          value := (value * minus_one);
        end if;
        return value;
      end random_ED_integer;

-- Initialization of package
use calendar;
begin
  init_value := integer(seconds(clock));
  if (init_value > 0) then
    init_value := -1 * init_value;
  end if;
end random;
\end{verbatim}
