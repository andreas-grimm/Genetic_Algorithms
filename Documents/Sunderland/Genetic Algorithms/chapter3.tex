In the first two chapters I had the focus on very simple problems, which do
not come up with too many problems. The major problem in the area of genetic
algorithms is that most of the problems are not so simple. So it is necessary to
improve the mechanisms and to define some more additional functions.

In this chapter I start with the most possible problems and show, how they are
handled in the common literature.
\section{Constraints}
By using genetic algorithms at more complex problems, the system cannot provide
a bijective relation between the search space and the binary string anymore.
In this cases an injective relation is used. This means, that there exist strings
without a representation in the search space.

Of course, these strings must not enter the next generation. In other words,
these strings have a fitness value of 0.

This becomes complicated, if the number of zeroed individuals is so big, that
the remaining population is too small to produce an new healthy generation. In
this case the whole system is not properly working and there is no simple possibility
to repair it. One very small chance to save the system is provided by the concept
of the handicapped individuals, but if this does not work, the fitness function
must be redefined.

By defining more and more complex problems the probability of constraints destroying
the system increases. So constraints should only be used with greatest caution.
If possible, they should be always avoided.
\section{The Method Of Fitness Scaling}
The most familar effect in the use of genetic algorithms is that the optimization
in the first phase in the run does not happens or is very slow. If this happens it
could have two reasons.
\begin{enumerate}
  \item The problem is not suitable for genetic algorithms, or
  \item The results of the individuals i.e. the fitness does not show big
differences.
\end{enumerate}
It is very difficult to prove that a problem is not suitable, and this could
not be said just after the first experiments. The second conclusion is more
reasonable.

If the difference of the fitness is not big enough, the programmer should spread
up this difference artificially. This means, that the difference even between
two individuals with nearly the same properties makes a difference in the reproduction.

By this way it is possible to use genetic algorithms at problems were normally
they would not work.

\subsection{Spreading the Fitness}
The method of fitness scaling is very simple, but also very effective. The first
step is to substract the fitness of the worst individual from the fitness of
the whole generation, i.e. from every single individual.

Having done this, the generation now has a fitness between 0 and a certain $x$.
But the difference between the individuals is as big as before. The next step
is simply to multiply the fitness of the individuals by a fixed number. Now the
difference is wide enough to find significant differences.

By reinspecting the fitness the algorithm will find a possibility to select between
the individuals, so that the optimization has a better change to work properly.
\subsection{Using the Fitness Scaling}
Because the problems with the fitness are common the fitness scaling is used
very often. This means that the programmer of the genetic algorithm has to choose
between two possibilities using this mechanism :
\begin{enumerate}
  \item Constantly - the fitness of the generation is always spreaded and checked.
But this also has the result, that the user cannot check, if there is really
an optimization.
  \item Only if needed - so that if the difference between the single individuals
is big enough, then there will no fitness scaling. This gives the user the possibility
to check, if the optimization over the generations is really effective.
\end{enumerate}
Normally the second way of the fitness scaling is used. Only in some very extreme
cases the first one will be choosen.
\section{The Generation Gap}
An unwanted effect in the use of genetic algorithms is the exist of phase 2 and
phase 3 as shown in figure \ref{PhasesOfCurves}. A very proper behaviour of the 
curves would be, if they stay up and
do not swing on a lower level. For the curve of the best values this is very easy
to realize. But the method has no parallel in the world of nature.

The simplest method would be not to destroy the best individuum in the next
cycle. Even if this will slow down the speed of optimization it also keep the
best individuals alive. The way doing this is to reduce the number of the new
individuals in the next generation by the size of the generation gap. The spare
space in the next generation is filled by the best individuals of the present
generation. The size of the gap is keeped variable, so that it can be changed
in the experiments. But it should not be greater then a third of the generations
size. Otherwise it is to dominant and the reproduction of generation does not
have a proper effect. Best results alway occur, when the size of the gap is
small, in the area of one to three individuals. By using the size like this,
there is still the effect of the generation gap, but the slow down of the
optimization is not to big.

The slow down of the optimization happens, because the genetic material of the
best individuals does not join the generation of the best generation. This means,
that the curve of the best individual does not go down, but also that the rest
of the population must produce a better individual to replace the present gap.
At the end of phase one this becomes more and more difficult and after the end
of phase one normally only a mutation does change the gap.

A better way of the gap is also to eliminate the individual with the worst fitness
and replace it with the individual with the best fitness. On this way it is
possible to keep the best individual alive and also to bring the genetic material
in the circle of reproduction.
\begin{figure}
  \unitlength1.0cm
  \begin{picture}(15,15)
  \end{picture}
\caption[The Effect of the Generation Gap]{The Effect of the Generation Gap}
\label{GenerationGap}
\end{figure}
Then the effect of the gap is reduced on the best individual and the rest of
the population can work on as it would do without the generation gap.
\section{Fixed Point Crossover}
Some problems works with values, which can't be simply transformed into a binary
string. These problems are not suitable for a simple genetic algorithm. So a
mechanism must be developed to work with these problems.

This mechanism is called fixed point crossover. This means, that the crossover
will not take place at a randomly selected point, but at some predefined single
points in the string.
These points do not necessary be in a specific order. They only must to be defined
by the exercise to be solved with the genetic algorithm.

This mechanism helps to put binary digits together to integer numbers, so that
also strings of integer numbers can be used within genetic algorithms.

By defining methods for the algorithms it is important to see, if there is a
equivalent in nature. This mechanism has such an equivalent. A natural genetic
string consists of more than to bases, i.e. a human genetic string consists
of four different bases. This means, that we do not have a binary system. The
single bases do not change in the crossover. The only thing happens is, that they
are recombined in a different order.

To simulate this the string in the computer must not be cut at a randomly defined
point. Only every second point in the string may be used for the crossover. So
the paired genes of the string form an integer value from 0 to 3, which could
be an equivalent to the four bases in nature.

As far as known, this mechanism is invented by the dutch scientist Lucansius\cite{Luc91}
which made experiments to analyse to human DNA with the help of genetic algorithms.
Even these kind of systems are not very common, it is good to notify that a
solution to these problems is possible and approved.
\section{The Handicapped Individual}
A big problem by the use of genetic algorithms is when the defined constraints
are too tough. Then the next generation does not contain enough surviving individuals
to produce a healthy population. Mostly the genetic algorithms stop, because
the is insufficent genetic material to generate even a new generation.

Another reason to keep these individuals is the possibility that the information
provided in this individual is very important and only a little defect makes
this special individual sort out off the population.

These effect occurs very often. When it is not possible to redefine the fitnessfunction
the user has to find an other way to keep the system running. A method to do
this is the method of the handicapped individual\cite{Gri91b}. First experiments
with a good result are made by Jochum et al\cite{Joc91}.

Individuals, which are sorted out are called {\bf Totgeburten}. These {\bf Totgeburten}
could even provide important informations, which should not get lost. So by
changing the behaviour of the constraints it is possible to keep the individual
in the population. As a fact, the method does not change the constraint but the
individual. This changed individual is called a handicapped individual.

\begin{definition}[A Handicapped Individual] is a {\bf Totgeburten}, which can
stay in the population if there exist a correct individual, so that the distance
between these individuals is not greater then a certain number and a function,
which transforms the {\bf Totgeburten} to the correct individual.
\end{definition}

This definition can help to define a correcting function. To define this function
it is necessary to define the distance between two individuals. I used this
distance in the definition above.

\begin{definition}[The Distance] of two individuals is defined by
$$x,y \in {\cal I}: x \ominus y : \sum_{i = 1}^{lchrom} | x_i - y_i | $$
\end{definition}
With this it is possible to define the correcting function:
\begin{definition}[The Correcting Function] is defined by
$$k({\cal I}): \forall x: x \notin {\cal I} \to x' \in {\cal I}$$
and also the following condition must be satisfied:
$$x' \ominus x \Longrightarrow min$$
\end{definition}

Additionaly there is a value $\xi$ which defines a cut. A individual with more
malicious informations than this number $\xi$ must not be repaired.
\subsection{Differences To The Normal Constraints}
Normal constraints are normally destructive, they eliminate individuals if they
do not fit in the environment. This does not happens in nature, in which there
are handicapped individuals at all. The problem of these handicapped individuals
is that they have a lot of problems to survive.

If they do it, they are very tough and have a quite good genetic material. This
genetic material could be necessary to keep the whole species alive. So the
concept of handicapped individuals is quite close to nature and experiments
\cite{Joc91} have shown, that genetic algorithms work more properly if the
constraints are replaced by the concept of handicapped individuals.
\section{Diploid and Haploid Crossover}
In nature there are two different possibilities of exchanging information between
two generations. In the chapters before I discribed a simple mechanism of cutting
the string with the genetic information and exchanging the parts. This is called
a halpoid crossover, according to the term in biology. But there is a different
way to generate new information.

This way is called the diploid crossover. It works simular to the way a crossover
with the DNA is done. The genetic string of both individuals are melting together
to a new string and then this new string doubles to build the information of
both new individuals.

This looks more like the way nature does the crossover, but in the area of genetic
algorithms this is not very much in use. The problem of this system is, that
it makes the whole genetic algorithm very complicated.

Another danger of this crossover is that the number of different individuals is
artificially decreased. The programmer should be very carefully to use this
mechanism.

Because of the habits of this method I like to advice the user to consult the
book of Goldberg\cite{Gol89} before installing this procedure into his program.