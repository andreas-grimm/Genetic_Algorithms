A very new development in the area of genetic algorithms is the introduction
of two dimensional genetic algorithms. This chapter will show the reasons which
leads to the development of these kind of algorithms and the way these are
working. The last part of this chapter is a brief discription of a system developed
to solve some problems with two dimensional genetic algorithms. The theory of
genetic algorithms is not discussed in the public, so this chapter is a short
discription of the research which is done so far.

\section{The Theory of Two Dimensional Genetic Algorithms}
The idea of the use of two dimensional genetic algorithms was born by trying
to optimize a system where the subject of the optimization could easily be transformed
into a matrix. But transforming this into a string looked quite more complicated.
In this specific problem the task was to optimize a structure from the graph theory.

As known a graph could be represented by a matrix, where a conection between
the $m$th and the $n$ knot is represented by a '1' at the $m$th row and the
$n$th column and a '1' at the $n$th row and the $m$th column of this matrix.
This matrix is called the adjacenz matrix of the graph.

This representation is very simple and has its fundaments in the mathematical
graph theory. A different representation into a binary string will be much more
complicated.

If it is possible to prove the equivalence between an one dimensional and a two
dimensional individual it should be possible to define the basic operations of
genetic algorithms this way that they are suitable for two dimensions.

\subsection{Equivalence between 1-D- and 2-D-Individuals}
The equivalence between the two different types of individuals can easily be shown
by defining a function which projects an arbitrary $m \times n$-matrix on a
$k \times 1$-matrix. The size $k$ of the second matrix is clearly defined by
the multiplication of $m$ and $n$ : $k = m \cdot n$.

The second matrix can be interpreted as an binary string. This is exactly the
form needed to work with genetic algorthms. If the dimensions of the matrix are
fixed\footnote{as they are usually} the inversion is defined, too.

So the first way of handling two-dimensional structures would be the following
algorithm :
\begin{enumerate}
  \item check the fitness of the individual
  \item transform the individual into a string
  \item use the rules for selection, crossover and mutation as known
  \item inverse the transformation
  \item go to the first step
\end{enumerate}
This algorithm works properly, the only reason to think about another possibility
is of philosophical nature.

Is there a satisfactional reason to transform the two-dimensional individual ?
And are there matrices where a transformation could destroy the structure of
the individual ? By taking the example above it could be seen, that there cannot
be any function to transform this individual which should be prefered to use.
With other words: how should this transformation be made ? Is it nescessary to
concat the rows of the matrix to construct the string, or should it be a concatination
of the columns, or should it be a quite different function ?

It is possible to pass this questions, if equivalent functions for selection,
crossover and mutation can be found. There would not be any reason to think
about this problems and the algorithms could be normally used.

\subsection{Equivalence of the Operations}
It is very simple to prove equivalence between the operations {\it selection},
because it is not nescessary to define an operation different from the one-dimensional
case. The structure of the individual is not important at this stage of the
reproduction.

There is no problem at the mutation, too. This operation only changes single
genes. This is independent from the structure of the individual. Only the number
of existing genes is important, so that an expansion to higher dimensions will
not change the behaviour of this operation.

A real difference can be found at the operation {\it crossover}.Here the structure
of the individual is important and there has to be an expansion of the definition
for one-dimensional individuals.

A proper definition of the two dimensional crossover would be the following
one :\\
Additional to the point of crossover, which is an intersection of either the
rows or the coloumns it is useful to define a second point which intersects the
other dimension of the individual. By this it is possible to define a cut in
the rows and the columns. The result of this is a matrix divided into four parts
or quadrants. By exchanging the first and the third quadrant between the partners
of the crossover it is possible to expand this operation to the new kind of
individuals.

Experiments with the new defined operations\cite{Gri91} show, that they are working
and producing sufficent results.
\section{Stability}
A theoretical term I did not refer so far is the {\it stability} of a string.

Fundamentically an individual is a binary string $A$, consisting of $k$ single
elements $a_1$ to $a_k$. The order of these elements\footnote{or genes} is not
important. To work with the operations of the genentic algorithms we need a full
population. This population is called {\bf A}$(t)$, where $t$ is the index for
the actual generation. Also a string $A'$ which is a part of the string $A$.
This string $A'$ consists of $k'$ genes.

Now the probability that the string $A'$ is distroyed while the crossover is
$\frac{k' - 1}{k}$. The possibility that the point of crossover is outside of
the string $A'$ is defined by the following formula
$$p_{string} = \frac{(k-1)-(k'-1)}{k-1} + c = \frac{k - k' - 2}{k-1} + c$$
(by \cite{Gol89}).
Additionally it has to be calculated that the string $A'$ is distroyed by the
mutation. This is the additional factor $c$ in the formula above. Explicit this
factor $c$ is defined by
$$c = (k - k') \cdot p_{mut}$$

To calculate the same values for two dimensional individuals it is necessary to
add a second part to this formula. The probability that a submatrix is distroyed
by the reproduction is defined by
$$p_{matrix} = p_{column} \cdot p_{row}$$

A submatrix $M'$ is a $k' \times k''$-Matrix inside of the matrix $M$. Because
the two points of crossover are selected randomally I can multiply the probabilities
for each dimension. The full formula is now given by
$$p_{matrix} = \(\frac{k - k' - 2}{k - 1} + (k - k') \cdot p_{mut}\) \cdot
               \(\frac{k - k'' - 2}{k - 1} + (k - k'') \cdot p_{mut}\)$$

Because both parts of this formula are always smaller then 1 it could be said
that a submatrix of $n$ elements is more stabil then a substring of $n$ elements.

\section{Demonstation of the Use of a 2D-GA}
This last section shows, how a genetic algorithm is used to optimize a graph\cite{Gri91a}.
Because the use of two dimensional individuals is new I do not have enough
literature to refer to. The results presented here were made by doing own researches.
The used package is documented in the appendix, it is mainly based on the AGA
package.

A graph is a mathematical structure. In this specific case it is represented
by an adjacence matrix. This binary matrix is by definition a $n \ times n$
matrix, where $n$ is the number of used knots in the graph. A definition of this
matrix is given in the section above.

A fitness function for graphs can be defined on many different ways. One of the
possible fitness functions is the {\it average path length}. The {\it average
path length} is a value which discripes the number of steps in a tree from the
root of the tree to a certain knot.

If the {\it average path length} is small a used search algorithms only needs
a few steps to reach the wanted knot. A full description of the theory is given
by {\sc Knuth}\cite{Knu68}.

Expanding this definition for arbitrary graphs gave a very good fitness function
and some experiments should show, how a graph will respond to this fitness function.

The surprising result of the tests was, that commonly the graph developed a very
compact structure so that the distance between single knots was very small.
This development was done in only a few generations, a fitness close to ninety
percent of the possible optimum was reached within 30 generations\footnote{based
on the experimental environment, using between 20 and 200 knots}.

The verify the results the experiments were repeated with a changed fitness
function. This time a function was used which could be called a {\it cost function}.
The function gives each possible connection between two knots a value. This value is
called the {\it cost} of the connection. The accumulated costs of the system
should be minimal.

This problem is typical for a class of problems occuring in the field of operation
research. It discripes the problem of finding the cheapest connection between
$A$ and $B$, for example the cheapest transport between two towns.

The results of the tests done here are simular to the results of the first
test series. The shape of the resulting graph look different from the first
one, it is not compact but more like line connecting the root to the edge with
the cheapest path.

Other experiments use 2D-GAs for indexing documents within an information retreival
system\cite{Loe92}. Results here show again, that this way of using genetic algorithms
work properly and produces quite good results.

It is important to say that these experiments are not final results of a research
with the aim to establish 2D-GAs into the theory, but single, summerized results.
It is nescessary to define experiments which confirm these results.
\section{Ideas for a further usage of 2D-GA}
The results made by the very first experiments are not satisfactionary enough
to think on serious use of this kind of genetic algorithms. But if the results
are confirmed by others they could be used in a wide area of subjects.

For example the optimization of neuronal networks can be done by using 2D-GAs.
At the California Institute of Technologie (CALTECH) experiments are made to
optimize those networks by genetic algorithms, but how these algorithms are
programmed is not known now.

A problem in the area of operations research is the transport of goods in a
factory such that the costs are minimal. These systems are very complex and
the algorithms to solve this problems are complicated. The time used to produce
plans for the transport can be reduced by using genetic algorithms.